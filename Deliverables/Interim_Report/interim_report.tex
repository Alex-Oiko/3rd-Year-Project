\documentclass[12pt,a4paper,twocolumn]{scrartcl}
\usepackage[latin1]{inputenc}
\usepackage{amsmath}
\usepackage{amsfonts}
\usepackage{amssymb}
\usepackage{textcomp}
\usepackage{cite}
\usepackage{graphicx} 
\author{Alexandros-Panagiotis Oikonomou}
\title{Interim Report}
\subtitle{Solving the Conjugate Gradient Method in a SpiNNaker Chip}
\begin{document}
\maketitle
\newpage
\thispagestyle{empty}
\mbox{}
\newpage
\thispagestyle{empty}
\mbox{}
\newpage
\thispagestyle{empty}
\mbox{}
\bf{Abstract. SpiNNaker is an asynchronous, event-driven parallel architecture designed to simulate the human brain. It has been designed to operate as a large scale neural in real-time network using a System-on-Chip multicore system. Its architecture is different from usual parallel computers, by only using	spikes to communicate, with no notion of sequential programming existing. That way it avoids usual pitfalls of parallel comouting, such as race conditions and deadlocks.  So far the most prominent uses of this architecture have been in neuroscience and robotics. The aim of this project is to put into use SpiNNaker's architecture by bringing it closer to classic computer science problems. The given problem is the conjugate gradient method, an iterative way of solving particular systems linear equations.}
\newline
\newline
\bf{Keywords.}\textnormal{SpiNNaker,parallel computing,supercomputers,linear algebra,conjugate gradient method,asynchronous,event-driven}
\section{Introduction}
\textnormal{
The SpiNNaker machine is an architecture inspired by the biolgy of the human brain. It can use up to\cite{Furber}} 
\section{Project Description}
\subsection{Neuron functionality}
\subsection{SpiNNaker Architecture}
\subsection{Conjugate Gradient Method}
\subsection{Final Description and Goals}
\section{Background Research and Literature}
\section{Final Design}
\section{Justification for the Approach}
\section{Completed Work}
\section{Plan for remaining work}
\section{Conclusion}
\section{References}
\bibliography{bibliog}{}
\bibliographystyle{plain}
\end{document}